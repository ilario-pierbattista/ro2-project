% !TEX root=index.tex
\chapter{Analisi dei risultati}\label{ch:cap_5}
	Gli esperimenti sono stati condotti con dei dati di istanza che fossero il più possibile realistici. Sono state analizzate tre classi di problemi: a flotta unitaria e senza time window, a flotta multipla senza time window ed infine a flotta multipla con time window.
	
	Le seguenti istanze sono state risolte utilizzando un laptop equipaggiato con una CPU Intel i3 quadcore a 2,67GHz e 4Gb di RAM (non più di 10 minuti di calcolo ad istanza). È stata utilizzata la versione 12.6 di CPLEX mantenendo le impostazioni di default per la risoluzione di problemi di PLIM (\emph{Programmazione Lineare Intera Mista}).

	\section{Presentazione dati statici ed ipotesi esemplificative}
	\label{sec:il_linguaggio_di_modellazione_ampl}
		Qui di seguito viene riportata una tabella (tabella \ref{table:parametri statici}) riassuntiva con tutti i valori dei parametri statici utilizzati nella formulazione delle istanze del modello:

		\begin{table}[]
		\centering
		\caption{Parametri dei veicoli e delle emissioni}
		\label{table:parametri statici}
			\begin{tabular}{@{}clc@{}}
				\toprule
				Notazione    & \multicolumn{1}{c}{Descrizione}                  & Valore \\ \midrule
				$g$          & costante gravitazionale (m/$s^2$)                & 9,81   \\
				$\rho$       & densità dell'aria (kg/$m^3$)                     & 1,255  \\
				$c_f$        & costo del carburante (€/l)                       & 1,20   \\
				$c_e$        & costo delle emissioni $CO_2$ (€/kg)              & 0,034  \\
				$U_{diesel}$ & energia intrinseca in 1 litro di carburante (MJ) & 36,4   \\
				$f_e$        & emissioni di $CO_2$ (kg/l)                       & 2,65   \\
				$\eta$       & efficienza del motore diesel (\%)                & 45     \\
				$M$          & massa a pieno carico del veicolo (kg)            & 8000   \\
				$w$          & massa a vuoto del veicolo (kg)                   & 4400   \\
				$C_r$        & coefficiente di resistenza al rotolamento        & 0,02   \\
				$C_d$        & resistenza alla penetrazione dell'aria           & 0,7    \\
				$A$          & area frontale del veicolo ($m^2$)                & 7      \\
				$p$          & salario dell'autista (€/h)                       & 8,5    \\ \bottomrule
			\end{tabular}
		\end{table}