% !TEX root=index.tex
\chapter{Descrizione formale del Pollution Routing Problem}\label{ch:cap_2}

Il \emph{Pollution Routing Problem}, come già accennato, è un’estensione del classico problema del \emph{Vehicle Routing} (vedi Capitolo \ref{ch:introduzione}) con una più ampia e comprensiva funzione obiettivo. Il problema consiste nell’instradare una flotta di veicoli per servire un set di clienti entro determinati intervalli temporali, calcolando la loro velocità in ogni tratto di strada e andando a minimizzare una funzione che tiene conto del carburante consumato, delle emissioni e del costo degli autisti. 

Il PRP può essere rappresentato formalmente tramite la teoria dei grafi: si prende un grafo completo $G = (\mathcal{N},\mathcal{A})$ dove $\mathcal{N} = \{1, 2, \cdots, n\}$ è l'insieme di nodi, $0$ è il deposito, $\mathcal{N}_0 = \mathcal{N} \backslash \{0\}$ è il set di clienti ed $\mathcal{A}$ è l'insieme di archi che collegano una coppia di nodi.