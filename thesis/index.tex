\documentclass[a4paper,11pt,oneside]{book}

\setcounter{tocdepth}{4} % IMPOSTA LA PROFONDITÀ DEGLI INDICI NELLA TABELLA DEI CONTENUTI
\setcounter{secnumdepth}{4} % IMPOSTA LA PROFONDITÀ DEGLI INDICI PER LE SEZIONI


\usepackage{packages}
\usepackage{frontespizio}
\usepackage{directives}

\counterwithout{figure}{chapter} % IMPOSTA LA NUMERAZIONE CONTINUA DELLE FIGURE, ALTRIMENTI VIENE FATTA PER CAPITOLI

%\makeindex                                          %CREA INDICE ANALITICO


%Document begin
\begin{document}
	\usemintedstyle{xcode} %IMPOSTA LO SCHEMA COLORI PER IL CODICE
	\setminted{linenos,autogobble} %IMPOSTAZIONI FORMATTAZIONE CODICE
	\pagenumbering{roman} %IMPOSTA NUMERAZIONE ROMANA FINO ALL'INTRODUZIONE
	\afterpage{\cfoot{\thepage}} %METTE I NUMERI DI PAGINA IN FONDO AL FOOTER. NECESSARIO SE SI USA FANCYHDR
	\frontespizio
	\clearpage{\pagestyle{empty}\cleardoublepage}
	\tableofcontents
	\newpage
	\listoffigures % INDICE DELLE FIGURE
	\listoflistings % INDICE DEL CODICE
	\newpage
	\pagenumbering{arabic} %IMPOSTA NUMERAZIONE TRADIZIONALE A PARTIRE DALL'INTRODUZIONE
	\pagebreak
	\setlength{\parskip}{1em} % IMPOSTA QUANTO SPAZIO LASCIO UN INVIO
	
	%===========================
	\input{"cap_0"}
	%===========================
	%===========================
	\input{"appendici"}
	%===========================
	
	%Bibliografia
	\bibliographystyle{alpha}
	\bibliography{bibliografia} %N.B. LA BIBLIOGRAFIA NON VIENE MOSTRATA FIN QUANDO NON VIENE RICHIAMATO
								%	  ALMENO UNA VOLTA IL RIFERIMENTO NEL TESTO
	\addcontentsline{toc}{chapter}{Bibliografia}
\end{document}
