% !TEX root=index.tex
\clearpage{\pagestyle{empty}\cleardoublepage}
\chapter{Introduzione}\label{ch:introduzione}
\markboth{Introduzione}{Introduzione}

I trasporti, essenziali nella nostra quotidianità allo svolgimento della maggior parte delle attività, hanno un impatto ambientale estremamente significativo, causato, soprattutto, dai veicoli su gomma. Il grande dispendio di energia, che deriva per lo più dal consumo di petrolio, contribuisce quotidianamente ad alimentare problemi come l’eccessivo sfruttamento di risorse, l’utilizzazione del suolo, l’acidificazione dei terreni e degli oceani, nonché tutti gli effetti indotti dalle emissioni dei gas serra\footnote{I gas serra (in inglese, \emph{Greenhouse Gas}) sono quei gas presenti in atmosfera, che sono trasparenti alla radiazione solare in entrata sulla Terra, ma riescono a trattenere, in maniera consistente, la radiazione infrarossa emessa dalla superficie terrestre, dall'atmosfera e dalle nuvole. Questa loro proprietà causa il cosiddetto effetto serra.}.

Le emissioni di questi gas, in particolare quelle di CO2, sono le più preoccupanti poiché hanno effetti sulla salute umana sia diretti (es.: l’inquinamento dell’aria) che indiretti (es.: il buco nell’ozono). Nel 2010, il 90\% della popolazione urbana della maggior parte dei paesi  dell’Unione Europea ha vissuto in aree al limite del valore europeo di qualità dell’aria per livelli di ossido di azoto, monossido di carbonio e di benzene \cite{effects-of-pollution}. \\*
Nel 2013, il trasporto su strada ha contribuito per più della metà del monossido di carbonio e ossidi di azoto, e per più di un quarto degli idrocarburi emessi nell’aria \cite{trucks-pollution}.

I veicoli ed i carburanti \emph{puliti} rientrano sicuramente tra le soluzioni che forniscono un metodo accessibile e significativo per la riduzione dell’inquinamento dell’aria e dei cambiamenti climatici generati dal settore dei trasporti su strada. Questi sicuramente includono veicoli a basso consumo di energia, carburanti ecologici che producono meno emissioni, così come auto e autocarri elettrici che eliminino interamente il problema delle emissioni dei gas di scarico.

Una completa riqualificazione dell’apparato logistico, sia in contesti statali che privati, è ancora un’operazione troppo costosa, nonché una mossa incompatibile con l’attuale rete di approvvigionamento per i veicoli a carburante ecologico ed elettrici, sia in scala nazionale che europea. \\*
Per questo motivo, al fine di fronteggiare le preoccupazioni sempre crescenti riguardo il pericoloso effetto dei trasporti sull’ambiente, si è iniziato a rivedere la pianificazione dei trasporti su strada, andando a tenere conto degli aspetti negativi prodotti dall’inquinamento.

L’obiettivo di questa tesina è quello di analizzare una variante del problema del Vehicle Routing, chiamato \emph{Pollution Routing Problem} (PRP) e introdotto soltanto di recente in letteratura, il quale tiene conto oltre a tutti i parametri classici, anche dell’inquinamento nell’instradamento dei veicoli.